% Options for packages loaded elsewhere
\PassOptionsToPackage{unicode}{hyperref}
\PassOptionsToPackage{hyphens}{url}
%
\documentclass[
]{article}
\usepackage{lmodern}
\usepackage{amssymb,amsmath}
\usepackage{ifxetex,ifluatex}
\ifnum 0\ifxetex 1\fi\ifluatex 1\fi=0 % if pdftex
  \usepackage[T1]{fontenc}
  \usepackage[utf8]{inputenc}
  \usepackage{textcomp} % provide euro and other symbols
\else % if luatex or xetex
  \usepackage{unicode-math}
  \defaultfontfeatures{Scale=MatchLowercase}
  \defaultfontfeatures[\rmfamily]{Ligatures=TeX,Scale=1}
\fi
% Use upquote if available, for straight quotes in verbatim environments
\IfFileExists{upquote.sty}{\usepackage{upquote}}{}
\IfFileExists{microtype.sty}{% use microtype if available
  \usepackage[]{microtype}
  \UseMicrotypeSet[protrusion]{basicmath} % disable protrusion for tt fonts
}{}
\makeatletter
\@ifundefined{KOMAClassName}{% if non-KOMA class
  \IfFileExists{parskip.sty}{%
    \usepackage{parskip}
  }{% else
    \setlength{\parindent}{0pt}
    \setlength{\parskip}{6pt plus 2pt minus 1pt}}
}{% if KOMA class
  \KOMAoptions{parskip=half}}
\makeatother
\usepackage{xcolor}
\IfFileExists{xurl.sty}{\usepackage{xurl}}{} % add URL line breaks if available
\IfFileExists{bookmark.sty}{\usepackage{bookmark}}{\usepackage{hyperref}}
\hypersetup{
  pdftitle={Exponential distribution is interval between consecutive Poisson events},
  pdfauthor={Dan Li},
  hidelinks,
  pdfcreator={LaTeX via pandoc}}
\urlstyle{same} % disable monospaced font for URLs
\usepackage[margin=1in]{geometry}
\usepackage{graphicx}
\makeatletter
\def\maxwidth{\ifdim\Gin@nat@width>\linewidth\linewidth\else\Gin@nat@width\fi}
\def\maxheight{\ifdim\Gin@nat@height>\textheight\textheight\else\Gin@nat@height\fi}
\makeatother
% Scale images if necessary, so that they will not overflow the page
% margins by default, and it is still possible to overwrite the defaults
% using explicit options in \includegraphics[width, height, ...]{}
\setkeys{Gin}{width=\maxwidth,height=\maxheight,keepaspectratio}
% Set default figure placement to htbp
\makeatletter
\def\fps@figure{htbp}
\makeatother
\setlength{\emergencystretch}{3em} % prevent overfull lines
\providecommand{\tightlist}{%
  \setlength{\itemsep}{0pt}\setlength{\parskip}{0pt}}
\setcounter{secnumdepth}{-\maxdimen} % remove section numbering

\title{Exponential distribution is interval between consecutive Poisson
events}
\author{Dan Li}
\date{2020-08-24}

\begin{document}
\maketitle

Let's denote the interval between consecutive Poisson events with random
variable Y, during the interval that extends from a to a + y, the number
of Poisson events k has the probability
\(P(k)=e^{-\lambda y} \frac{(\lambda y)^k}{k!}\), if \(k=0\),
\(e^{-\lambda y}\frac{(\lambda y)^0}{0!}=e^{-\lambda y}\) means there is
no event during the (a,a+y) time period.

Because there will be no occurrences in the interval (a, a + y) only if
Y \textgreater{} y, so \(P(Y > y)=e^{-\lambda y}\), then the cdf is
\(F_Y(y)=P(Y \le y)=1-P(Y > y)=1-e^{-\lambda y}\). Then the pdf is
\(\frac{d}{dy}F_Y(y)=\lambda e^{-\lambda y}, \quad y>0\), which is a
Exponential random variable.

The moment-generating function for a Exponential random variable Y is:
\[
\begin{align}
M_Y(t)=E(e^{tY})&=\int_{0}^{+\infty}e^{ty}\lambda e^{-\lambda y}dy\\
&=\lambda\int_{0}^{+\infty}e^{-(\lambda-t)y}dy\\
&=-\frac{\lambda}{\lambda-t}e^{-(\lambda-t)y}\Big|_{y=0}^{y\to+\infty}\\
&=\frac{\lambda}{\lambda-t}
\end{align}
\]

The expected value is:
\(M_Y^{(1)}(t)=\frac{d}{dt}\frac{\lambda}{\lambda-t}=\frac{\lambda}{(\lambda-t)^2}|_{t=0}=1/\lambda=E(Y)\)

\(M_Y^{(2)}(t)=\frac{2\lambda}{(\lambda-t)^3}|_{t=0}=2/\lambda^2=E(Y^2)\)
Then, the Variance
is:\(Var(Y)=E(Y^2)-(E(Y))^2=2/\lambda^2-1/\lambda^2=1/\lambda^2\)

\end{document}
